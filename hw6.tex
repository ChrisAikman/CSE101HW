\documentclass[12pt]{article}
\usepackage{graphicx}
\usepackage{hyperref}
\usepackage{cite}
\usepackage{float} % for [H] anchoring method
\usepackage{mdframed}
\usepackage{scrextend}
\usepackage{pgffor} % \foreach
\usepackage[top=1.4in, bottom=1.4in, left=1.4in, right=1.4in]{geometry}
\usepackage{tocloft}

\usepackage{listings}

\lstset{numbers=left}


\graphicspath{{pngs/}}

\begin{document}

\begin{titlepage}
\begin{center}
\Huge {CSE 101 HW \#6}\\
[1cm]
\normalsize

Chris Aikman\\
Hugo Rivera\\
[1cm]

{\today}

\end{center}
\end{titlepage}

\newpage

\section{Programming Paradigm Assignment}
\subsection{Imperative}
\begin{addmargin}[1cm]{0em}
\textbf{Example Language:}
	\begin{addmargin}[1cm]{0em}
		Java
	\end{addmargin}
\end{addmargin}

\begin{addmargin}[1cm]{0em}% 1em left, 2em right
\textbf{Application:}
	\begin{addmargin}[1cm]{0em}
		Java has been used in thousands of embedded devices, but it may be most commonly known these days for near exclusivity of programming on Android devices. Almost all applications on the Android mobile OS have been programmed with Java.\\
		Android SDK: \url{https://developer.android.com/sdk/index.html}
	\end{addmargin}
\end{addmargin}

\begin{addmargin}[1cm]{0em}
\textbf{Application Appropriateness:}
	\begin{addmargin}[1cm]{0em}
		Java was an appropriate choice for designing embedded systems as well as applications for the Android OS because of the large support for the language, as well as its ability to be easily written and read. Java focuses on how states are changed in a program which makes it reliable when debugging and also makes the code easy to read. Code in Java is designed to be modular with emphasis placed on code structure so that programmers joining a project can quickly look through the code and get an understanding of how it works.
	\end{addmargin}
\end{addmargin}

\subsection{Functional}
\begin{addmargin}[1cm]{0em}
\textbf{Example Language:}
	\begin{addmargin}[1cm]{0em}
		Haskell
	\end{addmargin}
\end{addmargin}

\begin{addmargin}[1cm]{0em}
\textbf{Application:}
	\begin{addmargin}[1cm]{0em}
		Haskell has been used to program several web servers, most notably Snap and Yesod.\\
		Snap: \url{http://snapframework.com/}\\
		Yesod: \url{http://www.yesodweb.com/}
	\end{addmargin}
\end{addmargin}

\begin{addmargin}[1cm]{0em}
\textbf{Application Appropriateness:}
	\begin{addmargin}[1cm]{0em}
		Haskell was an appropriate choice for designing these applications because it offers a powerful type system that protects the programmer from making trivial mistakes while still being flexible enough for real world use.
	\end{addmargin}
\end{addmargin}

\newpage
\subsection{Object Oriented}
\begin{addmargin}[1cm]{0em}
\textbf{Example Language:}
	\begin{addmargin}[1cm]{0em}
		C++
	\end{addmargin}
\end{addmargin}

\begin{addmargin}[1cm]{0em}
\textbf{Application:}
	\begin{addmargin}[1cm]{0em}
		C++ is well known for being the language chosen to implement the two most common web browsers: Google Chrome and Mozilla Firefox.\\
		Google Chrome: \url{http://www.google.com/chrome/}\\
		Firefox: \url{https://www.mozilla.org/en-US/firefox/new/}\\
	\end{addmargin}
\end{addmargin}

\begin{addmargin}[1cm]{0em}
\textbf{Application Appropriateness:}
	\begin{addmargin}[1cm]{0em}
		C++ was an appropriate choice for these web browsers because it gives the programmer a lot of power while offering the advantages of object oriented programming, which allows code to be smaller, more modular and easier to manage. C++ has also been used, tested and evolved for decades making it extremely reliable when used correctly. 
	\end{addmargin}
\end{addmargin}

\subsection{Logic}
\begin{addmargin}[1cm]{0em}
\textbf{Example Language:}
	\begin{addmargin}[1cm]{0em}
		Prolog
	\end{addmargin}
\end{addmargin}

\begin{addmargin}[1cm]{0em}
\textbf{Application:}
	\begin{addmargin}[1cm]{0em}
	 Prolog has been successfully used for many years in the creation of artificial intelligence simulations. Most notably, NASA used Prolog to create a voice-operated procedure browser for astronauts.\\
	 Programming AI:\url{http://www.pearsoned.co.uk/highereducation/resources/bratkoprologprogrammingforartificialintelligence3e/}
	 NASA's Clarissa: \url{http://ti.arc.nasa.gov/tech/cas/user-centered-technologies/clarissa/}
	\end{addmargin}
\end{addmargin}

\begin{addmargin}[1cm]{0em}
\textbf{Application Appropriateness:}
	\begin{addmargin}[1cm]{0em}
		Prolog was an appropriate choice for designing artificial intelligence simulations because its focus is on using rules and facts to generate solutions to problems. The logic programming paradigm that Prolog follows makes solving complicated problems that are common to artificial intelligence much easier than thinking of how to solve the issues using a different approach.
	\end{addmargin}
\end{addmargin}




\section{Parallel Matrix Multiply Assignment}
\subsection{Pseudocode}
Haskell is a declarative language, thus this is declarative pseudocode.

\begin{verbatim}
a matrix is a 2D array of elements that can be multiplied and added

to multiply two matrices a (dimensions n by m), b (dimensions m by p):
    let the result be matrix c of dimensions
    let m = shared dimension of b and a
            if there is no shared dimension, throw an error

    map (ra ->
        map (cb ->
            sum [ra[i]*b[i] | i<-[0..m]])
        b.cols)
    a.rows
\end{verbatim}

\subsection{Haskell Code}
\lstinputlisting[language=Haskell]{matrix_mult.hs}
\subsection{Assembly Code Computation}
Done using ``ghc -S matrix\_mult.hs'' See the appendix.
\subsection{Assembly Code Multiplication}
This happens in line 52 after preparing the arguments.
Haskell performs a jump the number library's integer multiplication
method ``jmp base\_GHCziNum\_zt\_info'' (instructions for Base.GHC.zahltimes or
integer mult)
\subsection{Number of Assembly Code Lines for Multiplication}
It takes 6 lines of code starting at line 47 to gather the arguments and
prepare the program for jumping to ``base\_GHCziNum\_zt\_info''

\section{Appendix}
\lstinputlisting{matrix_mult.s}

\end{document}
