\documentclass[hidelinks,12pt]{article}
\usepackage{graphicx}
\usepackage{hyperref}
\usepackage{cite}
\usepackage{float} % for [H] anchoring method
\usepackage{mdframed}
\usepackage{pgffor} % \foreach
\usepackage[top=1.4in, bottom=1.4in, left=1.4in, right=1.4in]{geometry}
\usepackage{tocloft}

\graphicspath{{pngs/}}

\renewcommand*\contentsname{Table of Contents}
\renewcommand{\cftsecleader}{\cftdotfill{\cftdotsep}}

\begin{document}

\begin{titlepage}
\begin{center}
\Huge {CSE 101 HW \#6}\\
[1cm]
\normalsize

Chris Aikman\\
Hugo Rivera\\
[1cm]

{\today}

\end{center}
\end{titlepage}

\newpage

\section{Programming Paradigm Assignment}
\subsection{Imperative}
\subsubsection{Example Language}
\subsubsection{Application}
\subsubsection{Application Appropriateness}
\subsection{Functional}
\subsubsection{Example Language}
Haskell is an example of a functional language.
\subsubsection{Application}
Haskell has been used to program X
\subsubsection{Application Appropriateness}
\subsection{Object Oriented}
\subsubsection{Example Language}
C++ is an example of an object oriented language.
\subsubsection{Application}
\subsubsection{Application Appropriateness}
\subsection{Logic}
\subsubsection{Example Language}
\subsubsection{Application}
\subsubsection{Application Appropriateness}

\section{Parallel Matrix Multiply Assignment}
\subsection{Pseudocode}
\subsection{Haskell Code}
\subsection{Assembly Code Computation}
\subsection{Assembly Code Multiplication}
\subsection{Number of Assembly Code Lines for Multiplication}

\end{document}