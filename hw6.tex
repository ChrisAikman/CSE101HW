\documentclass[hidelinks,12pt]{article}
\usepackage{graphicx}
\usepackage{hyperref}
\usepackage{cite}
\usepackage{float} % for [H] anchoring method
\usepackage{mdframed}
\usepackage{scrextend}
\usepackage{pgffor} % \foreach
\usepackage[top=1.4in, bottom=1.4in, left=1.4in, right=1.4in]{geometry}
\usepackage{tocloft}

\usepackage{listings}

\lstset{numbers=left}


\graphicspath{{pngs/}}

\begin{document}

\begin{titlepage}
\begin{center}
\Huge {CSE 101 HW \#6}\\
[1cm]
\normalsize

Chris Aikman\\
Hugo Rivera\\
[1cm]

{\today}

\end{center}
\end{titlepage}

\newpage

\section{Programming Paradigm Assignment}
\subsection{Imperative}
\begin{addmargin}[1cm]{0em}% 1em left, 2em right
\textbf{Example Language:}
	\begin{addmargin}[1cm]{0em}% 1em left, 2em right
		XXX
	\end{addmargin}
\end{addmargin}

\begin{addmargin}[1cm]{0em}% 1em left, 2em right
\textbf{Application:}
	\begin{addmargin}[1cm]{0em}% 1em left, 2em right
		XXX
	\end{addmargin}
\end{addmargin}

\begin{addmargin}[1cm]{0em}% 1em left, 2em right
\textbf{Application Appropriateness:}
	\begin{addmargin}[1cm]{0em}% 1em left, 2em right
		XXX
	\end{addmargin}
\end{addmargin}

\subsection{Functional}
\begin{addmargin}[1cm]{0em}% 1em left, 2em right
\textbf{Example Language:}
	\begin{addmargin}[1cm]{0em}% 1em left, 2em right
		Haskell
	\end{addmargin}
\end{addmargin}

\begin{addmargin}[1cm]{0em}% 1em left, 2em right
\textbf{Application:}
	\begin{addmargin}[1cm]{0em}% 1em left, 2em right
		Haskell has been used to program
	\end{addmargin}
\end{addmargin}

\begin{addmargin}[1cm]{0em}% 1em left, 2em right
\textbf{Application Appropriateness:}
	\begin{addmargin}[1cm]{0em}% 1em left, 2em right
		XXX
	\end{addmargin}
\end{addmargin}

\subsection{Object Oriented}
\begin{addmargin}[1cm]{0em}% 1em left, 2em right
\textbf{Example Language:}
	\begin{addmargin}[1cm]{0em}% 1em left, 2em right
		C++
	\end{addmargin}
\end{addmargin}

\begin{addmargin}[1cm]{0em}% 1em left, 2em right
\textbf{Application:}
	\begin{addmargin}[1cm]{0em}% 1em left, 2em right
		XXX
	\end{addmargin}
\end{addmargin}

\begin{addmargin}[1cm]{0em}% 1em left, 2em right
\textbf{Application Appropriateness:}
	\begin{addmargin}[1cm]{0em}% 1em left, 2em right
		XXX
	\end{addmargin}
\end{addmargin}

\subsection{Logic}
\begin{addmargin}[1cm]{0em}% 1em left, 2em right
\textbf{Example Language:}
	\begin{addmargin}[1cm]{0em}% 1em left, 2em right
		XXX
	\end{addmargin}
\end{addmargin}

\begin{addmargin}[1cm]{0em}% 1em left, 2em right
\textbf{Application:}
	\begin{addmargin}[1cm]{0em}% 1em left, 2em right
		XXX
	\end{addmargin}
\end{addmargin}

\begin{addmargin}[1cm]{0em}% 1em left, 2em right
\textbf{Application Appropriateness:}
	\begin{addmargin}[1cm]{0em}% 1em left, 2em right
		XXX
	\end{addmargin}
\end{addmargin}




\section{Parallel Matrix Multiply Assignment}
\subsection{Pseudocode}
Haskell is a declarative language, thus this is declarative pseudocode.

\begin{verbatim}
a matrix is a 2D array of elements that can be multiplied and added

to multiply two matrices a (dimensions n by m), b (dimensions m by p):
    let the result be matrix c of dimensions
    let m = shared dimension of b and a
            if there is no shared dimension, throw an error

    map (ra ->
        map (cb ->
            sum [ra[i]*b[i] | i<-[0..m]])
        b.cols)
    a.rows
\end{verbatim}

\subsection{Haskell Code}
\lstinputlisting[language=Haskell]{matrix_mult.hs}
\subsection{Assembly Code Computation}
Done using ``ghc -S matrix\_mult.hs'' See the appendix.
\subsection{Assembly Code Multiplication}
This happens in line 52 after preparing the arguments.
Haskell performs a jump the number library's integer multiplication
method ``jmp base\_GHCziNum\_zt\_info'' (instructions for Base.GHC.zahltimes or
integer mult)
\subsection{Number of Assembly Code Lines for Multiplication}
It takes 6 lines of code starting at line 47 to gather the arguments and
prepare the program for jumping to ``base\_GHCziNum\_zt\_info''

\section{Appendix}
\lstinputlisting{matrix_mult.s}

\end{document}
